% DOCUMENT FORMATING
\documentclass[12pt]{article}
\usepackage[margin=1in]{geometry}

% PACKAGES
\usepackage{amsmath} % For extended formatting
\usepackage{amssymb} % For math symbols
\usepackage{amsthm} % For proof environment
\usepackage{array} % For tables
\usepackage{enumerate} % For lists
\usepackage{extramarks} % For headers and footers
\usepackage{blindtext}
\usepackage{fancyhdr} % For custom headers
\usepackage{graphicx} % For inserting images
\usepackage{multicol} % For multiple columns
\usepackage{verbatim} % For displaying code
\usepackage{tkz-euclide}	
\usepackage{pgfplots}
\newtheorem{theorem}{Theorem}[section]
\newtheorem*{theorem*}{Theorem}
\newtheorem{corollary}{Corollary}[theorem]
\newtheorem{lemma}[theorem]{Lemma}
\usepackage[Algorithm]{algorithm}
\usepackage[noend]{algpseudocode}
% SET UP HEADER AND FOOTER
\pagestyle{fancy}
\lhead{\MyCourse} % Top left header
\chead{\MyTopicTitle} % Top center header
\rhead{\MyAssignment} % Top right header
\lfoot{\MyCampus} % Bottom left footer
\cfoot{} % Bottom center footer
\rfoot{\MySemester} % Bottom right footer
\renewcommand\headrulewidth{0.4pt} % Size of the header rule
\renewcommand\footrulewidth{0.4pt} % Size of the footer rule
% ----------
% TITLES AND NAMES 
% ----------

\newcommand{\MyCourse}{ICS 311, Sadowski, Section 1}
\newcommand{\MyTopicTitle}{Problem Set 07}
\newcommand{\MyAssignment}{Taylor D. Gabatino}
\newcommand{\MySemester}{Fall 2020}
\newcommand{\MyCampus}{University of Hawaii at Manoa}
\begin{document}
\subsection*{1. Constructing and Extracting the Solution for Longest Path(10 pts)}
\linebreak
In the following you use the posted solution to LongestPathValueMemoized from class last week. \\
\linebreak
\textbf{(a)} (3 pts) Rewrite LongestPathValueMemoized to LongestPathMemoized that takes an additional parameter \textbf{next[1...$\lvert$V$\rvert$]]}, and records the next vertex in the path from any given vertex u in the next[u]. Assume that all entries of next are initialized to 1 by the caller. \\
\linebreak
\begin{center}
\includegraphics[scale=0.5]{1a.png}\\
\end{center}
\linebreak
\textbf{(b)} (3 pts) Once that is done, write a procedure that recovers (e.g., prints) the path from s to t by tracing through next. \\
\linebreak
\begin{center}
\includegraphics[scale=0.7]{1b.png} \\
\end{center}
\linebreak
\textbf{(c)} (2 pts) What is the asymptotic runtime of your total solution to the Longest Path problem in terms of $\lvert$V$\rvert$ and $\lvert$E$\rvert$? Include all steps (initializing arrays, LongestPathMemoized, and printing the solution). Hint: Count in aggregate across all calls rather than trying to figure out how many times the loop runs on a given Adj[u]. \\
\linebreak
The asymptotic runtime of the total solution is $\theta$($\lvert$E$\rvert$). This is due to the traversal of the edges. \\
\linebreak
\textbf{(d)} (2 pts) What is the asymptotic use of space of your solution in terms of $\lvert$V$\rvert$ and $\lvert$E$\rvert$?  \\
\linebreak
The asymptotic use of space in the solution is $\theta$($\lvert$V$\rvert$ + $\lvert$E$\rvert$)\\
\subsection*{2. LCS by Suffix(15 pts)}
\linebreak
In this problem you will redo the derivation of the LCS dynamic programming algorithm to be a variation that works on the suffixes rather than the prefixes. The intention is that revising a published derivation will help you see the approach before we ask you to do one "from scratch" on your own. \\
\linebreak
In Topic 12 Lecture Notes Dynamic Programming \\
$X_i$ = prefix $<x1,...,xi>$\\
$Y_i$ = prefix $<y1,...,yi>$\\
\linebreak
But the optimal LCS substructure(s) can also be suffixes rather than prefixes. (Informally: It doesn't matter whether we start looking for the substructures from the beginning of the sequences X and Y of from their end.) In the suffix approach the arrows go in a more natural direction. \\
\linebreak
Below we step you through the derivation of a symmetric version of LCS based on suffixes: \\
\linebreak
$X_i$ = suffix $<xi,...,xm>$\\
$Y_j$ = suffix $<yi,...,ym>$\\
\linebreak
\textbf{(a)} (3 pts) Reformulate \textbf{Theorum 15.1} for the suffix version by filling in the blanks below: \\
\linebreak
Let Z = $<z_1,...,z_k>$ be any LCS of X = $<x_1,...,x_m>$ and Y = $<y_1,...,y_n>$. Then: \\
\linebreak
1. If $x_1 = y+1$, this means that $x_1 = y_1 = z_1$. This means it is an LCS of $X_2$ and $Y_2$. \\
2. (If the first characters of X and Y match, then these first characters are also the first character of the LCS Z, so we can discard the first character of all three and continue recursively on the suffix.) \\
3. If $x_1 \neq y_1$ this means that $z_1 \neq x_1$. This means that Z is an LCS of $X_2$ and Y. \\
4. If $x_1 \neq y_1$, this means that $z_1 \neq y_1$. This means that Z is an LCS of X and $Y_2$. \\
5. (If the first characters of X and Y don't match each other, the the suffix Z must be in the substrings not involving these characters, and furthermore we can use the first character of Z to determine which one it lies in.) \\
\linebreak
(To keep this problem set from getting too long we will skip the proof, but it is an easy translation of the proof on page 392 of CLRS.) \\
\linebreak
\textbf{(b)} (5 pts) Now redefine the \textbf{Recursive formulation} accordingly (again, fill in the blanks). \\
\linebreak
Define c[i,j] = length of LCS of $X_i$ and $Y_j$. We want to find c[,]. \\
  \begin{equation}
    X=
    \begin{cases}
      0, & \text{if}\ i > m || j > n \\
      c[i+1] j+1, & \text{if}\ i \leqslant m, j \leqslant n && x_i = y_j \\
      max(c[i,j+1],c[i+1,j]), & \text{if}\ i \leqslant m, j \leqslant n && x_i \neq y_i
    \end{cases}
  \end{equation}
\textbf{(c)} (5 pts) Write pseudocode for LCS-LENGTH according to your \textbf{Recursive formulation}. (Hint: the arrows need to be different.) \\
\linebreak
\begin{center}
\includegraphics[scale=0.5]{2x.png} \\
\end{center}
\linebreak
*Got help from Matthew Kirts in Section 01. \\
\linebreak
\textbf{(d)} (2 pts) In the Notes, the longest subsequence could only be "printed" with PRINT-LCS and the pseudocode needed recursion. With your "suffix" method, you can do better and simplier: Write LCS(b,X,m,n) pseudocode that returns the subsequence directly. Use a vector to store the result. A vector is like an array but grows as needed. \\
\linebreak
\begin{center}
\includegraphics[scale=0.5]{2d.png}\\
\end{center}
\linebreak
*Got help from Matthew Kirts in Section 01. \\
\subsection*{3. Activity Scheduling with Revenue (15 pts)} 
\linebreak
Activity scheduling problem from the greedy algorithm class: Suppose that different activities earn different amounts of revenue. In addition to their start and finish times $s_i$ and $f_i$, each activity $a_1$ has a revenue $r_i$ and our objective is now to \textbf{maximize the total revenue}: \\
\linebreak
\begin{center}
\includegraphics[scale=0.99]{Equation.png} \\
\end{center}
In class you found out that we can't use a greedy algorithm to maximize the revenue from activities, and we noted that dynamic programming will apply. Here you will develop the DP solution following the same steps as for the other problems (e.g., problem 3 above), but you are responsible for the details. Your analysis in (a) and (b) below should mirror that of section 16.1 of CLRS. \\
\linebreak
\textbf{(a)} (3 pts) Describe the structure of an optimal solution for $A_{ij}$ for $S_{ij}$, as defined in CLRS and use a cut and paste argument to show that the problem has optimal substructure. \\
\linebreak
By referring to the cut and paste argument in CLRS, we need to show the optimal solution for $A_{ij}$ has to be the optimal solution for $S_{ik}$ and $S_{kj}$. In order to do this, we would find the set of compatible activities in $S_{kj}$ where the revenue is maximized.  In other words, the optimal solution $A_{ij}$ would consist of the sub components $A_{ik}$, $a_k$, and $A_{kj}$. \\
\linebreak
\textbf{(b)} (3 pts) Write a recursive definition of the value val[i,j] of the optimal solution for $S_{ij}$. \\
\linebreak
  \begin{equation}
    X=
    \begin{cases}
      0, & \text{if}\ S_{ij} = 0 \\
      max(val[i,k] + val[k,j] + r_k), & \text{if}\ S_{ij} \neq 0 \\
    \end{cases}
  \end{equation}
  \linebreak
\textbf{(c)} (6 pts) Translate that definition into pseudocode that computes the optimal solution. Hints: Create fictitious activities where $a_0$ with $f_0 = 0$ and $a_{n+1}$ with $s_{n+1 = \infty}. $ Define tables val[0...n+1, 0...n+1], where activity[i,j] is the activity $k$ that is chosen for $A_{ij}$. Use a bottom-up approach, filling in the tables for smallest problems first and then increasing difference of j-i. \\
\linebreak
\begin{center}
\includegraphics[scale=0.75]{3x.png} \\
\end{center}
\linebreak
\textbf{(d)} (2 pts) Write pseudocode to print out the set of activities chosen. \\
\linebreak
\begin{center}
\includegraphics[scale=0.75]{3d.png} \\
\end{center}
\linebreak
\textbf{(e)} (1 pt) What is the asymptotic runtime of your solution including (c) and (d)? \\
\linebreak
The asymptotic runtime of this solution is $O(n^2)$, due to the nested for loops and iteration through each one. 
\end{document}